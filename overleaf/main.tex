\documentclass[a4paper, 12pt]{article}

% Packages
\usepackage[utf8]{inputenc}  % Encoding
\usepackage[margin=1in]{geometry}  % Page margins
\usepackage{amsmath}  % Math symbols
\usepackage{graphicx}  % Include graphics
\usepackage{hyperref}  % Hyperlinks
\usepackage{caption}  % Custom captions for figures/tables

% Title Information
\title{An Example LaTeX Document}
\author{Your Name}
\date{\today}

\begin{document}

% Title Page
\maketitle

% Abstract
\begin{abstract}
This document serves as an example of a basic \LaTeX\ file structure. It demonstrates sections, figures, tables, and references to help users create their documents efficiently.
\end{abstract}

% Introduction Section
\section{Introduction}
This is an example of an introduction section. \LaTeX\ is widely used in academic and professional settings due to its high-quality typesetting capabilities.

% Equations
\section{Mathematical Expressions}
Here is an example of an inline equation: \( E = mc^2 \).

A displayed equation:
\[
\int_a^b f(x)\,dx = F(b) - F(a)
\]

% Figures
\section{Figures}
Figure \ref{fig:example} shows a placeholder image.

\begin{figure}[h]
    \centering
    \includegraphics[width=0.5\textwidth]{example-image} % Replace with your image file
    \caption{This is an example figure.}
    \label{fig:example}
\end{figure}

% Tables
\section{Tables}
Table \ref{tab:example} shows an example table.

\begin{table}[h]
    \centering
    \begin{tabular}{|c|c|c|}
    \hline
    Column 1 & Column 2 & Column 3 \\
    \hline
    A & B & C \\
    X & Y & Z \\
    \hline
    \end{tabular}
    \caption{An example table.}
    \label{tab:example}
\end{table}

% References
\section{References}
References can be added using \texttt{bibtex}. For now, here is a manual example:
\begin{itemize}
    \item Knuth, D. E. (1984). \textit{The TeXbook}. Addison-Wesley.
    \item Lamport, L. (1994). \textit{\LaTeX: A Document Preparation System}. Addison-Wesley.
\end{itemize}

% Conclusion
\section{Conclusion}
This document provides a simple template to get started with \LaTeX. Customize it as per your requirements.

\end{document}
